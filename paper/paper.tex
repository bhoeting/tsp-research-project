\documentclass{article}

\usepackage[left=3cm,right=3cm,top=2cm,bottom=2cm]{geometry}
\usepackage[fleqn]{amsmath}
\usepackage{amssymb}
\usepackage{mathtools}
\usepackage{enumerate}
\usepackage{clrscode3e}
\usepackage{parskip}
\usepackage{mathptmx}

\documentclass[12pt]{report}

\begin{document}

\title{Exploring the Traveling Salesman Problem}
\author{Brennan Hoeting}
\date{}
\maketitle 

\section*{Abstract}
This paper will explore the Traveling Salesman Problem.
We will first obtain an understanding of the problem, that
is, we will explain the problem, examine its input and output,
and identify what makes solving this problem particularly
challenging.  Next, we will present a naive brute force solution
to the problem and analyze it's efficiency and correctness.  We
will improve upon that solution by comparing and contrasting
more efficient approaches to solving the problem.  We will choose
an efficient solution to present, explain, and compare runtimes
against naive solution.  We will prove that our efficient solution
can solve a subclass of instances of the Traveling Salesman Problem.
Finally, Python code for the naive and efficient solutions will be
presented and demonstrated.

\section*{Understanding the Problem}
The traveling salesman problem can be described as follows:
Given a set of cities, where there is a distance between
each pair of cities, find a tour one could take that 
begins at a starting city, visits each destination city
exactly once, and returns to the starting city with the
minimum possible total distance traveled. \par

The problem input is a matrix that defines the distance
between each pair of cities.  We will denote this as
the distance matrix $D_{ij}$, where 


\section*{A Naive Brute Force Solution}


  
\end{document}
